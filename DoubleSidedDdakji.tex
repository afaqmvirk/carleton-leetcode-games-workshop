\documentclass[12pt]{article}
\usepackage{amsmath}
\usepackage{amssymb}
\usepackage{fancyhdr}
\usepackage{hyperref}

\pagestyle{fancy}
\fancyhf{}
\fancyhead[L]{Challenge 1.0}
\fancyhead[R]{ddakji Challenge}

\begin{document}

\title{\textbf{LeetCode Challenge: ddakji}}
\author{}
\date{}
\maketitle

\section*{Problem Description}

A \textit{ddakji} is a grid with four symbols drawn on its top face, arranged as follows:

\[
\begin{array}{cc}
A & B \\
C & D \\
\end{array}
\]

The bottom face of the ddakji is blank. The ddakji can be flipped \textbf{vertically} or \textbf{horizontally}, as described below:

\begin{itemize}
    \item A \textbf{vertical flip} (\texttt{V}) is peformed as shown:
    \[
    \begin{array}{cc}
    A & B \\
    C & D \\
    \end{array}
    \xrightarrow{\texttt{V}}
    \begin{array}{cc}
    C & D \\
    A & B \\
    \end{array}
    \]
    \item A \textbf{horizontal flip} (\texttt{H}) is performed as shown:
    \[
    \begin{array}{cc}
    A & B \\
    C & D \\
    \end{array}
    \xrightarrow{\texttt{H}}
    \begin{array}{cc}
    B & A \\
    D & C \\
    \end{array}
    \]
\end{itemize}

The ddakji starts in its original orientation:
\[
\begin{array}{cc}
A & B \\
C & D \\
\end{array}
\]

You are given a string of flips, consisting of the characters \texttt{V} (vertical flip) and \texttt{H} (horizontal flip), such as \texttt{VVHVVHHHHHVVVVV}. 

\begin{enumerate}
    \item Determine the number of times the ddakji returns to its original orientation, including the starting position.
    \item Output the final orientation of the ddakji after all the flips.
\end{enumerate}

\section*{Input Format}
A single string $S$ of length $1 \leq |S| \leq 10^5$, consisting only of the characters \texttt{V} and \texttt{H}.

\section*{Output Format}
\begin{enumerate}
    \item An integer $N$, the number of times the ddakji is in its original orientation.
    \item The final orientation of the ddakji in a $2 \times 2$ grid format.
\end{enumerate}

\section*{Sample Input}
\texttt{VVHVVHHHHHVVVVV}

\section*{Sample Output}
\begin{verbatim}
6
C D
A B
\end{verbatim}

\section*{Explanation}
The sequence of flips alternates the orientation of the ddakji. After every flip, the state is checked to see if it matches the original orientation:
\[
\begin{array}{cc}
A & B \\
C & D \\
\end{array}
\]

By the end of the sequence, the ddakji has returned to its original orientation a total of $6$ times (including the initial position). The final orientation after all flips is:
\[
\begin{array}{cc}
C & D \\
A & B \\
\end{array}
\]

\section*{Constraints}
\begin{itemize}
    \item The input string contains only \texttt{V} and \texttt{H}.
    \item The grid always starts in the orientation shown above.
    \item The function must execute efficiently for large inputs.
\end{itemize}

\section*{Notes}
This problem tests your ability to simulate operations on a grid and recognize cyclical patterns.

\end{document}
